\documentclass[12pt]{article}
\usepackage[T1,T2A,T2B,T2C]{fontenc}
\usepackage[utf8]{inputenc}
\usepackage[english, russian]{babel}
\usepackage{amsmath,amssymb,amsthm,amscd,amsfonts,mathrsfs,amsthm}
\usepackage{array} 
\usepackage[a4paper,margin=1in,landscape]{geometry}
\newcommand*{\TitleParbox}[1]{\parbox[c]{2.40cm}{\raggedright #1}}

\newcolumntype{C}[1]{%
 >{\vbox to 5ex\bgroup\vfill\centering}%
 p{#1}%
 <{\egroup}} 
\begin{document}
\delimitershortfall=-1pt
Дана сигнатура 
$$\Sigma = \langle E^{(4)},S^{(2)};5^{(0)},3^{(0)},2^{(0)}\rangle,$$ 
множество 
\begin{multline*}
A=\{\text{номера группы}\}\cup\{\text{номера студенческих билетов}\}\cup
\{\text{имена преподавателей}\}\cup\{\text{коды предметов}\}\cup\{\text{оценки}\}
\end{multline*}
и алгебраическая система $\mathcal{A}=(\Sigma,A)$, где
\begin{table}[h]
\begin{tabular}{|c|m{3cm}|m{3cm}|m{3cm}|c|}
\hline 
$E_A$ & студенческий билет & номер предмета & имя преподавателя & оценка\\[2ex]
\hline 
 &  &  &  &  \\ 
\hline 
\end{tabular}$\quad$ 
\end{table}
\begin{table}[h]
\begin{tabular}{|c|m{3cm}|c|}
\hline 
$S_A$ & студенческий билет & номер группы \\[2ex]
\hline 
 &  &   \\ 
\hline 
\end{tabular}$\quad$ 
\end{table}

и $5_A, 3_A, 2_A$ - оценки.
\begin{enumerate}
\item Преподаватель ставит хотя бы по одной 5 в каждой группе.
\begin{multline*}
(\forall group)
(\forall ID)\Big(S(ID, group)\to (\exists ID)(\exists subject)(\exists grade)\big(S(ID, group)\wedge E(ID,subject,x,grade) \wedge grade \approx 5\big)\Big)
\end{multline*}
\item В каждой группе больше 1 но меньше 4 отличников.
\begin{multline*}
(\forall group)(\forall ID)\bigg(S(ID, group)\to(\exists A1)(\exists A2)\Big(A1 \not\approx A2 \wedge \\
(\exists subject)(\exists teacher)\big(E(A1,subject,teacher,5)\big)\wedge 
(\exists subject)(\exists teacher)\big(E(A2,subject,teacher,5)\big)\wedge \\
(\forall ID)\big(S(ID, group)\wedge(\exists subject)(\exists teacher)\big(E(ID,subject,teacher,5)\big) \to\\
 (ID\approx A1 \vee ID\approx A2)\big)\Big)\bigg)\\
\vee\\
(\forall group)(\forall ID)\bigg(S(ID, group)\to(\exists A1)(\exists A2)(\exists A3)\Big((A1 \not\approx A2) \wedge (A1 \not\approx A3) \wedge (A2 \not\approx A3) \wedge\\
(\exists subject)(\exists teacher)\big(E(A1,subject,teacher,5)\big)\wedge 
(\exists subject)(\exists teacher)\big(E(A2,subject,teacher,5)\big)\wedge \\
(\exists subject)(\exists teacher)\big(E(A3,subject,teacher,5)\big)\wedge \\
(\forall ID)\big(S(ID, group)\to (ID\approx A1 \vee ID\approx A2 \vee ID\approx A3)\big)\Big)\bigg)
\end{multline*}
\item Есть ровно 1 группа, в которой есть хотя бы 2 кандидата на отчисление. $F_s=1$, если $\sigma(id)$ - студент на отчисление, $F_s=0$ иначе.
\begin{multline*}
(\exists group)(\exists ID)\Bigg((S(ID, group)\wedge (\exists F1)(\exists F2)\big(S(F1,group)\wedge S(F1,group)\wedge(F1 \not\approx F2) \wedge
(\exists id)(id\approx F1 \wedge F_s)\wedge (\exists id)(id\approx F2 \wedge F_s)\big) \wedge\\
 \neg\bigg((\exists group2)(\exists ID)\Big(S(ID, group)\wedge group\not\approx group2 \wedge(\exists F1)(\exists F2)\big(S(F1,group2)\wedge S(F1,group2)\wedge(F1 \not\approx F2) \wedge\\
  (\exists id)(id\approx F1 \wedge F_s)\wedge (\exists id)(id\approx F2 \wedge F_s)\big) \Big)\bigg)\Bigg)
\end{multline*}
\item Нет групп, в которых не было бы хорошистов
\begin{multline*}
(\forall group)(\forall ID)(S(ID, group)\to (\exists ID)(S(ID, group)\wedge (\exists subject)(\exists teacher)(\exists grade)(E(ID,subject,teacher, grade)\wedge\\
 grade \not\approx 2 \wedge grade \not\approx 3 \wedge grade \not\approx 5 )))
\end{multline*}
\end{enumerate}
\end{document}