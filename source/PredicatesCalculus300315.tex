\input{glyphtounicode}
\pdfgentounicode=1
\documentclass[12pt,a4paper]{article}


\usepackage[a4paper,margin=1in, left=10mm, top=20mm, right=10mm, bottom=20mm, nohead, nofoot, landscape,paperheight=15in]{geometry}
\usepackage[T2A]{fontenc}
\usepackage[utf8]{inputenc}
\usepackage[english, russian]{babel}
\usepackage{amsmath,amsthm}
\newtheorem{theorem}{Theorem}
\newtheorem{lemma}{Лемма}
\newtheorem{proposition}{Утверждение}
\usepackage{bussproofs}
\pagenumbering{gobble}



\def\defaultHypSeparation{\hskip .1in}
\def\fCenter{\ \vdash\ }
\begin{document}
\begin{enumerate}
\item Вывести $\neg(\exists x)[\varphi \wedge (\exists z)\psi]\vee (\forall y)\theta \fCenter (\exists y)\theta \to (\forall x)[(\forall z)\neg \psi \vee \neg\varphi]$
\begin{prooftree}
				
				\AxiomC{}
			\UnaryInfC{$\neg(\exists x)[\varphi \wedge (\exists z)\psi]\vee (\forall y)\theta, (\exists y)\theta \fCenter (\forall z)\neg \psi \vee \neg\varphi$}
				\AxiomC{}
			\UnaryInfC{$\neg(\exists x)[\varphi \wedge (\exists z)\psi]\vee (\forall y)\theta, (\exists y)\theta \fCenter (\forall x)(\neg(\varphi \wedge (\exists x)\psi)) \to (\forall z)\neg \psi \vee \neg\varphi$}
		\BinaryInfC{$\neg(\exists x)[\varphi \wedge (\exists z)\psi]\vee (\forall y)\theta, (\exists y)\theta \fCenter (\forall z)\neg \psi \vee \neg\varphi$}
	\UnaryInfC{$\neg(\exists x)[\varphi \wedge (\exists z)\psi]\vee (\forall y)\theta, (\exists y)\theta \fCenter (\forall x)[(\forall z)\neg \psi \vee \neg\varphi]$}
\UnaryInfC{$\neg(\exists x)[\varphi \wedge (\exists z)\psi]\vee (\forall y)\theta \fCenter (\exists y)\theta \to (\forall x)[(\forall z)\neg \psi \vee \neg\varphi]$}
\end{prooftree}
Бесперспективно.
\begin{proposition}
Секвенция $\neg(\exists x)[\varphi \wedge (\exists z)\psi]\vee (\forall y)\theta \fCenter (\exists y)\theta \to (\forall x)[(\forall z)\neg \psi \vee \neg\varphi]$ не является выводимой.
\begin{proof}
Возьмём формулы $\varphi=x\approx x,\psi = z\approx z,\theta= y\approx y$ Тогда для любого состояния $\sigma$ верно $\sigma(\neg(\exists x)[\varphi \wedge (\exists z)\psi])=0$ и $\sigma((\forall y)\theta)=1$. Заметим, что 
\begin{enumerate}
\item $\neg(\exists x)[\varphi \wedge (\exists z)\psi]\equiv (\forall x)[(\forall z)\neg \psi \vee \neg\varphi]$, следовательно $\sigma((\forall x)[(\forall z)\neg \psi \vee \neg\varphi])=1$
\item $\sigma((\forall y)\theta)=1 \Rightarrow \sigma((\exists y)\theta)=1$
\end{enumerate}
Левая часть секвенции тождественно истинна. Правая часть секвенции тождественно ложна, так как посылка тождественно истинна и следствие тождественно ложно. Следовательно вся секвенция тождественно ложна.

Исчисление высказываний является непротиворечивым, следовательно секвенция $\neg(\exists x)[\varphi \wedge (\exists z)\psi]\vee (\forall y)\theta \fCenter (\exists y)\theta \to (\forall x)[(\forall z)\neg \psi \vee \neg\varphi]$ не является выводимой.
\end{proof}
\end{proposition}
\item Доказать что введение $\exists$ слева обратимо.
Сначала докажем, что $(\exists x)\varphi \fCenter \varphi$ выводимо:
\begin{prooftree}
	\AxiomC{$\varphi \fCenter \varphi$}
\RightLabel{\scriptsize(вв $\exists$ лев)}
\UnaryInfC{$(\exists x)\varphi \fCenter \varphi$}
\end{prooftree}
Теперь можно воспользоваться допустимым правилом:
\begin{prooftree}
	\AxiomC{$\Gamma, (\exists x)\varphi \fCenter \psi$}
\RightLabel{\scriptsize(доп. выв.)}	
\UnaryInfC{$\Gamma, \varphi \fCenter \psi$}
\end{prooftree}
\item Доказать что введение $\exists$ справа необратимо. Рассмотрим сигнатуру $\Sigma = (\leq^{(2)};0^{(0)},1^{(0)})$ и алгебраическую систему в этой сигнатуре: $\mathcal{A}=(\omega, \leq; 0, 1)$.

Пусть "введение $\exists$ справа" обратимое правило, тогда верно

\begin{prooftree}
	\AxiomC{$ \fCenter (\exists x) x\leq 0$}	
\UnaryInfC{$ \fCenter x\leq 0$}
\end{prooftree}
следовательно $\fCenter x\leq 0$ выводима и является тождественно истинной. Но $\fCenter x\leq 0$ не является тождественно истинной секвенцией, так как существует состояние $\sigma: \sigma(x)=1$, на котором формула $x\leq 0$ ложна.
\end{enumerate}


\end{document}