\input{glyphtounicode}
\pdfgentounicode=1
\documentclass[12pt,a4paper]{article}


\usepackage[a4paper,margin=1in, left=10mm, top=20mm, right=10mm, bottom=20mm, nohead, nofoot, landscape,paperheight=15in]{geometry}
\usepackage[T2A]{fontenc}
\usepackage[utf8]{inputenc}
\usepackage[english, russian]{babel}
\usepackage{amsmath,amsthm}
\newtheorem{theorem}{Theorem}
\newtheorem{lemma}{Лемма}
\usepackage{bussproofs}
\pagenumbering{gobble}



\def\defaultHypSeparation{\hskip .1in}
\def\fCenter{\ \vdash\ }
\begin{document}
\begin{enumerate}
\item Вывести $\neg(\exists x)[\varphi \wedge (\exists z)\psi]\vee (\forall y)\theta \fCenter (\exists y)\theta \to (\forall x)[(\exists z)\neg \psi \vee \varphi]$
\begin{prooftree}
				
				\AxiomC{}
			\UnaryInfC{$\neg(\exists x)[\varphi \wedge (\exists z)\psi]\vee (\forall y)\theta, (\exists y)\theta \fCenter (\forall z)\neg \psi \vee \neg\varphi$}
				\AxiomC{}
			\UnaryInfC{$\neg(\exists x)[\varphi \wedge (\exists z)\psi]\vee (\forall y)\theta, (\exists y)\theta \fCenter (\forall x)(\neg(\varphi \wedge (\exists x)\psi)) \to (\forall z)\neg \psi \vee \neg\varphi$}
		\BinaryInfC{$\neg(\exists x)[\varphi \wedge (\exists z)\psi]\vee (\forall y)\theta, (\exists y)\theta \fCenter (\forall z)\neg \psi \vee \neg\varphi$}
	\UnaryInfC{$\neg(\exists x)[\varphi \wedge (\exists z)\psi]\vee (\forall y)\theta, (\exists y)\theta \fCenter (\forall x)[(\forall z)\neg \psi \vee \neg\varphi]$}
\UnaryInfC{$\neg(\exists x)[\varphi \wedge (\exists z)\psi]\vee (\forall y)\theta \fCenter (\exists y)\theta \to (\forall x)[(\forall z)\neg \psi \vee \neg\varphi]$}
\end{prooftree}
\item Доказать что введение $\exists$ слева обратимо.
Сначала докажем, что $(\exists x)\varphi \fCenter \varphi$ выводимо:
\begin{prooftree}
	\AxiomC{$\varphi \fCenter \varphi$}
\RightLabel{\scriptsize(вв $\exists$ лев)}
\UnaryInfC{$(\exists x)\varphi \fCenter \varphi$}
\end{prooftree}
Теперь можно воспользоваться допустимым правилом:
\begin{prooftree}
	\AxiomC{$\Gamma, (\exists x)\varphi \fCenter \psi$}
\RightLabel{\scriptsize(доп. выв.)}	
\UnaryInfC{$\Gamma, \varphi \fCenter \psi$}
\end{prooftree}
\item Доказать что введение $\exists$ справа необратимо. Рассмотрим сигнатуру $\Sigma = (\leq^{(2)};+^{(2)},0^{(0)},1^{(0)})$ и алгебраическую систему в этой сигнатуре: $\mathcal{A}=(\omega, \leq; +, 0, 1)$.

Пусть "введение $\exists$ справа" - обратимое правило, тогда верно

\begin{prooftree}
	\AxiomC{$ \fCenter (\exists x) x\leq 0$}	
\UnaryInfC{$ \fCenter x\leq 0$}
\end{prooftree}
следовательно $\fCenter x\leq 0$ выводима и является тождественно истинной. Но $\fCenter x\leq 0$ не является тождественно истинной секвенцией, так как существует состояние $\sigma: \sigma(x)=1$, на котором формула $x\leq 0$ ложна.
\end{enumerate}


\end{document}